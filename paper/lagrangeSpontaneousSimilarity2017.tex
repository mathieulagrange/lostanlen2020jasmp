% -----------------------------------------------
% Template for ISMIR Papers
% 2017 version, based on previous ISMIR templates

% Requirements :
% * 6+n page length maximum
% * 4MB maximum file size
% * Copyright note must appear in the bottom left corner of first page
% * Clearer statement about citing own work in anonymized submission
% (see conference website for additional details)
% -----------------------------------------------

\documentclass{article}
\usepackage{amsmath,cite,url}
\usepackage{graphicx}
\usepackage{color}


% Title.
% ------
\title{Spontaneous Similarity Judgments of Musical Tones played with various Playing Techniques}

% Note: Please do NOT use \thanks or a \footnote in any of the author markup

% Single address
% To use with only one author or several with the same address
% ---------------
%\oneauthor
% {Names should be omitted for double-blind reviewing}
% {Affiliations should be omitted for double-blind reviewing}

% Two addresses
% --------------
%\twoauthors
%  {First author} {School \\ Department}
%  {Second author} {Company \\ Address}

%% To make customize author list in Creative Common license, uncomment and customize the next line
%  \def\authorname{First Author, Second Author}


% Three addresses
% --------------
\author{Mathieu Lagrange, Christian EL-Hajj, Mathias Rossignol, Grégoire Lafay}

%% To make customize author list in Creative Common license, uncomment and customize the next line
%  \def\authorname{First Author, Second Author, Third Author}

% Four or more addresses
% OR alternative format for large number of co-authors
% ------------
%\multauthor
%{First author$^1$ \hspace{1cm} Second author$^1$ \hspace{1cm} Third author$^2$} { \bfseries{Fourth author$^3$ \hspace{1cm} Fifth author$^2$ \hspace{1cm} Sixth author$^1$}\\
%  $^1$ Department of Computer Science, University , Country\\
%$^2$ International Laboratories, City, Country\\
%$^3$  Company, Address\\
%{\tt\small CorrespondenceAuthor@ismir.edu, PossibleOtherAuthor@ismir.edu}
%}
%\def\authorname{First author, Second author, Third author, Fourth author, Fifth author, Sixth author}


\sloppy % please retain sloppy command for improved formatting

\begin{document}
%
\maketitle
%
\begin{abstract}

Musical timbre is a multi-faceted notion that have been extensively studied, mostly focusing on its spectral aspect. By studying the spontaneous judgments of similarity between several instruments played with a diverse set of playing techniques, we aim in this paper at better understanding the influence of the change of the playing technique on the perception of timbre.

A first experiment is conducted by music experts to identify which couple of instrument and playing technique in a given dataset is worth comparing to other couples of instrument / playing technique. In a second experiment, spontaneous similarity judgments among the retained couples are collected using a canonic free sorting task experiment design.


To further study the outcomes of this experiment, we assume a two-step computational model of human perception, where the acoustic signal is fed to a statically designed processing unit that accounts for frequency and temporal modulations. The resulting features are then projected using a supervised technique which considers as input the perceptual similarity judgments gathered in the second experiment. Numerical experiments show that the induced perceptual space is able to approximate perceptual data with satisfying accuracy.

\end{abstract}
%
\section{Introduction}\label{sec:introduction}

Spontaneous similarity judgments

The contributions of the paper are as follows: 1) provide perceptual data that account for the perception of musical tones played with various playing techniques, 2), demonstrate for the gathered data, the type of playing technique plays an important role for the organization of the data, and 3) propose a perceptually motivated computational of timbre perception that account well for the gathered human judgments.

\section{Dimensions of Timbre}\label{sec:}

\section{Data}\label{sec:}

\section{Experiment 1}\label{sec:}

Problématique

Dans le cadre du projet TICEL, nous allons construire un algorithme permettant de mesurer les similarités entre différents modes de jeu. Par mode de jeu, nous entendons ici un couple "instrument + mode de jeu".

Afin de mesurer les performances de l'algorithme, nous avons besoins d'établir une vérité de terrain, en l’occurrence les similarités entre modes de jeu données par des humains.

Or il n'est actuellement pas envisageable de monter une expérience permettant d'établir cette vérité terrain. Le nombre de modes de jeu à comparer est trop important.

C'est pourquoi nous proposons une expérience intermédiaire ayant pour but de réduire le nombre de modes de jeu.

Objectif

Pour chaque mode de jeu, vous répondrez à la question suivante : Est il intéressant de comparer ce mode de jeu à un/pls autre(s) mode(s) d'un autre instrument

Procédure

Nous vous demandons de noter chaque mode de jeu avec un nombre d’étoiles. Plus le nombre d'étoiles est important, plus le mode de jeu est susceptible d'être associé de manière "analogique" à un autre mode de jeu d'un autre instrument. Pour exemple :

Une étoile : ce mode de jeu est singulier, il n’est pas utile de le comparer avec un autre mode de jeu d'un autre instrument.
Quatre étoiles : il y a bien une proximité sur un aspect du son entre ce mode de jeu et un autre mode d'un autre instrument, mais cette dernière n'est ni décisive ni évidente.
Sept étoiles : il a une grande proximité entre ce mode de jeu et un autre mode d'un autre instrument, il est donc intéressant de le conserver.
Déroulement

Vous pouvez réaliser l'expérience en plusieurs fois, mais veillez à noter l'ensemble des modes de jeu pour tous les instruments.

Usage

Votre notation doit prendre en compte l'ensemble des modes de jeu présents dans la base, bien que par souci de clarté, les modes de jeu soient présentés par instruments
Les modes de jeu déjà notés apparaîtront en vert
Pour chaque mode de jeu, vous pouvez écouter plusieurs enregistrements à différentes hauteurs et nuances



\section{Experiment 2}\label{sec:}

\section{Modeling}\label{sec:}


% For bibtex users:
\bibliography{bib}

% For non bibtex users:
%\begin{thebibliography}{citations}
%
%\bibitem {Author:00}
%E. Author.
%``The Title of the Conference Paper,''
%{\it Proceedings of the International Symposium
%on Music Information Retrieval}, pp.~000--111, 2000.
%
%\bibitem{Someone:10}
%A. Someone, B. Someone, and C. Someone.
%``The Title of the Journal Paper,''
%{\it Journal of New Music Research},
%Vol.~A, No.~B, pp.~111--222, 2010.
%
%\bibitem{Someone:04} X. Someone and Y. Someone. {\it Title of the Book},
%    Editorial Acme, Porto, 2012.
%
%\end{thebibliography}

\end{document}
